% Options for packages loaded elsewhere
\PassOptionsToPackage{unicode}{hyperref}
\PassOptionsToPackage{hyphens}{url}
%
\documentclass[
  ignorenonframetext,
]{beamer}
\usepackage{pgfpages}
\setbeamertemplate{caption}[numbered]
\setbeamertemplate{caption label separator}{: }
\setbeamercolor{caption name}{fg=normal text.fg}
\beamertemplatenavigationsymbolsempty
% Prevent slide breaks in the middle of a paragraph
\widowpenalties 1 10000
\raggedbottom
\setbeamertemplate{part page}{
  \centering
  \begin{beamercolorbox}[sep=16pt,center]{part title}
    \usebeamerfont{part title}\insertpart\par
  \end{beamercolorbox}
}
\setbeamertemplate{section page}{
  \centering
  \begin{beamercolorbox}[sep=12pt,center]{part title}
    \usebeamerfont{section title}\insertsection\par
  \end{beamercolorbox}
}
\setbeamertemplate{subsection page}{
  \centering
  \begin{beamercolorbox}[sep=8pt,center]{part title}
    \usebeamerfont{subsection title}\insertsubsection\par
  \end{beamercolorbox}
}
\AtBeginPart{
  \frame{\partpage}
}
\AtBeginSection{
  \ifbibliography
  \else
    \frame{\sectionpage}
  \fi
}
\AtBeginSubsection{
  \frame{\subsectionpage}
}
\usepackage{amsmath,amssymb}
\usepackage{lmodern}
\usepackage{iftex}
\ifPDFTeX
  \usepackage[T1]{fontenc}
  \usepackage[utf8]{inputenc}
  \usepackage{textcomp} % provide euro and other symbols
\else % if luatex or xetex
  \usepackage{unicode-math}
  \defaultfontfeatures{Scale=MatchLowercase}
  \defaultfontfeatures[\rmfamily]{Ligatures=TeX,Scale=1}
\fi
\usetheme[]{AnnArbor}
\usecolortheme{dolphin}
\usefonttheme{structurebold}
% Use upquote if available, for straight quotes in verbatim environments
\IfFileExists{upquote.sty}{\usepackage{upquote}}{}
\IfFileExists{microtype.sty}{% use microtype if available
  \usepackage[]{microtype}
  \UseMicrotypeSet[protrusion]{basicmath} % disable protrusion for tt fonts
}{}
\makeatletter
\@ifundefined{KOMAClassName}{% if non-KOMA class
  \IfFileExists{parskip.sty}{%
    \usepackage{parskip}
  }{% else
    \setlength{\parindent}{0pt}
    \setlength{\parskip}{6pt plus 2pt minus 1pt}}
}{% if KOMA class
  \KOMAoptions{parskip=half}}
\makeatother
\usepackage{xcolor}
\IfFileExists{xurl.sty}{\usepackage{xurl}}{} % add URL line breaks if available
\IfFileExists{bookmark.sty}{\usepackage{bookmark}}{\usepackage{hyperref}}
\hypersetup{
  pdftitle={IBV Transmission study to determine the transmission of pathogenic IBV (Challenge) among vaccinated Commercial Broilers compared to that of unvaccinated birds},
  pdfauthor={Egil A.J.Fischer},
  hidelinks,
  pdfcreator={LaTeX via pandoc}}
\urlstyle{same} % disable monospaced font for URLs
\newif\ifbibliography
\usepackage{color}
\usepackage{fancyvrb}
\newcommand{\VerbBar}{|}
\newcommand{\VERB}{\Verb[commandchars=\\\{\}]}
\DefineVerbatimEnvironment{Highlighting}{Verbatim}{commandchars=\\\{\}}
% Add ',fontsize=\small' for more characters per line
\usepackage{framed}
\definecolor{shadecolor}{RGB}{248,248,248}
\newenvironment{Shaded}{\begin{snugshade}}{\end{snugshade}}
\newcommand{\AlertTok}[1]{\textcolor[rgb]{0.94,0.16,0.16}{#1}}
\newcommand{\AnnotationTok}[1]{\textcolor[rgb]{0.56,0.35,0.01}{\textbf{\textit{#1}}}}
\newcommand{\AttributeTok}[1]{\textcolor[rgb]{0.77,0.63,0.00}{#1}}
\newcommand{\BaseNTok}[1]{\textcolor[rgb]{0.00,0.00,0.81}{#1}}
\newcommand{\BuiltInTok}[1]{#1}
\newcommand{\CharTok}[1]{\textcolor[rgb]{0.31,0.60,0.02}{#1}}
\newcommand{\CommentTok}[1]{\textcolor[rgb]{0.56,0.35,0.01}{\textit{#1}}}
\newcommand{\CommentVarTok}[1]{\textcolor[rgb]{0.56,0.35,0.01}{\textbf{\textit{#1}}}}
\newcommand{\ConstantTok}[1]{\textcolor[rgb]{0.00,0.00,0.00}{#1}}
\newcommand{\ControlFlowTok}[1]{\textcolor[rgb]{0.13,0.29,0.53}{\textbf{#1}}}
\newcommand{\DataTypeTok}[1]{\textcolor[rgb]{0.13,0.29,0.53}{#1}}
\newcommand{\DecValTok}[1]{\textcolor[rgb]{0.00,0.00,0.81}{#1}}
\newcommand{\DocumentationTok}[1]{\textcolor[rgb]{0.56,0.35,0.01}{\textbf{\textit{#1}}}}
\newcommand{\ErrorTok}[1]{\textcolor[rgb]{0.64,0.00,0.00}{\textbf{#1}}}
\newcommand{\ExtensionTok}[1]{#1}
\newcommand{\FloatTok}[1]{\textcolor[rgb]{0.00,0.00,0.81}{#1}}
\newcommand{\FunctionTok}[1]{\textcolor[rgb]{0.00,0.00,0.00}{#1}}
\newcommand{\ImportTok}[1]{#1}
\newcommand{\InformationTok}[1]{\textcolor[rgb]{0.56,0.35,0.01}{\textbf{\textit{#1}}}}
\newcommand{\KeywordTok}[1]{\textcolor[rgb]{0.13,0.29,0.53}{\textbf{#1}}}
\newcommand{\NormalTok}[1]{#1}
\newcommand{\OperatorTok}[1]{\textcolor[rgb]{0.81,0.36,0.00}{\textbf{#1}}}
\newcommand{\OtherTok}[1]{\textcolor[rgb]{0.56,0.35,0.01}{#1}}
\newcommand{\PreprocessorTok}[1]{\textcolor[rgb]{0.56,0.35,0.01}{\textit{#1}}}
\newcommand{\RegionMarkerTok}[1]{#1}
\newcommand{\SpecialCharTok}[1]{\textcolor[rgb]{0.00,0.00,0.00}{#1}}
\newcommand{\SpecialStringTok}[1]{\textcolor[rgb]{0.31,0.60,0.02}{#1}}
\newcommand{\StringTok}[1]{\textcolor[rgb]{0.31,0.60,0.02}{#1}}
\newcommand{\VariableTok}[1]{\textcolor[rgb]{0.00,0.00,0.00}{#1}}
\newcommand{\VerbatimStringTok}[1]{\textcolor[rgb]{0.31,0.60,0.02}{#1}}
\newcommand{\WarningTok}[1]{\textcolor[rgb]{0.56,0.35,0.01}{\textbf{\textit{#1}}}}
\usepackage{longtable,booktabs,array}
\usepackage{calc} % for calculating minipage widths
\usepackage{caption}
% Make caption package work with longtable
\makeatletter
\def\fnum@table{\tablename~\thetable}
\makeatother
\setlength{\emergencystretch}{3em} % prevent overfull lines
\providecommand{\tightlist}{%
  \setlength{\itemsep}{0pt}\setlength{\parskip}{0pt}}
\setcounter{secnumdepth}{-\maxdimen} % remove section numbering
\ifLuaTeX
  \usepackage{selnolig}  % disable illegal ligatures
\fi

\title{IBV Transmission study to determine the transmission of
pathogenic IBV (Challenge) among vaccinated Commercial Broilers compared
to that of unvaccinated birds}
\author{Egil A.J.Fischer}
\date{03-07-2022}

\begin{document}
\frame{\titlepage}

\begin{frame}
Source required code-files including loading the data
\end{frame}

\begin{frame}{Document structure}
\protect\hypertarget{document-structure}{}
\begin{itemize}
\tightlist
\item
  Estimation of the upperlimit of \(R\) in the vaccinated and lowerlimit
  of \(R\) in the unvaccinated birds (Velthuis et al.~2007).
\item
  Considerations for next trials
\end{itemize}

Based on previous discussions the ``M41'' strain is excluded.

For the unvaccinated groups in the ``GA08'' strain challenged group all
contact and seeder birds were positive at the first measurement. In the
vaccinated groups only a few seeders were positive, but none of the
contacts.
\end{frame}

\begin{frame}[fragile]{Final size}
\protect\hypertarget{final-size}{}
The final size are the number of animals that were infected during the
entire duration of an outbreak (or experiment). For small numbers the
exact distribution can be determined numerically for a known value of
\(R\). This can be used to determine the most likely value of \(R\) and
its boundaries given an observed final size. In case of no or all
contact animals being infected the most likely value is respectively 0
or \(\inf\) and only the upper- and lower boundary of the confidence
interval can be given.

\begin{verbatim}
## `summarise()` has grouped output by 'Group'. You can override using the
## `.groups` argument.
\end{verbatim}

\begin{longtable}[]{@{}llrrrrr@{}}
\caption{Input values for the final size calculations. fs = susceptibles
end of experiment, iS = contact birds beginning of experiment, iI =
challenged birds that excreed during experiment, iR = challenged birds
that do not excrete}\tabularnewline
\toprule
Group & Vaccinated & fs & iS & iI & iR & n \\
\midrule
\endfirsthead
\toprule
Group & Vaccinated & fs & iS & iI & iR & n \\
\midrule
\endhead
DMV 1639\_1\_1 & Yes & 1 & 10 & 10 & 0 & 20 \\
DMV 1639\_1\_2 & Yes & 0 & 10 & 7 & 3 & 20 \\
DMV 1639\_2\_1 & No & 9 & 10 & 10 & 0 & 20 \\
DMV 1639\_2\_2 & No & 10 & 10 & 10 & 0 & 20 \\
\bottomrule
\end{longtable}

\begin{block}{\(R\) with final size estimation}
\protect\hypertarget{r-with-final-size-estimation}{}
\begin{longtable}[]{@{}lrrrr@{}}
\caption{Estimate of \(R\) based on the final size. Estimate = best
value, 95\%-LL= lower limit,95\%-UL = upper limit, pval.above1 =
probability \(R\) is above 1}\tabularnewline
\toprule
Vaccinated & Estimate & 95\%-LL & 95\%-UL & pval.above1 \\
\midrule
\endfirsthead
\toprule
Vaccinated & Estimate & 95\%-LL & 95\%-UL & pval.above1 \\
\midrule
\endhead
Yes & 0.12 & 0.01 & 0.61 & 0.01 \\
No & 3.41 & 1.63 & 7.34 & 1.00 \\
\bottomrule
\end{longtable}
\end{block}

\begin{block}{\(R\) with final size estimation: Non excreting challenged
are S}
\protect\hypertarget{r-with-final-size-estimation-non-excreting-challenged-are-s}{}
\begin{longtable}[]{@{}lrrrr@{}}
\caption{Estimate of \(R\) based on the final size. Estimate = best
value, 95\%-LL= lower limit,95\%-UL = upper limit, pval.above1 =
probability \(R\) is above 1}\tabularnewline
\toprule
Vaccinated & Estimate & 95\%-LL & 95\%-UL & pval.above1 \\
\midrule
\endfirsthead
\toprule
Vaccinated & Estimate & 95\%-LL & 95\%-UL & pval.above1 \\
\midrule
\endhead
Yes & 0.10 & 0.01 & 0.54 & 0 \\
No & 3.41 & 1.63 & 7.34 & 1 \\
\bottomrule
\end{longtable}

\#Regression estimation of the transmission coefficient Using a
complementary log-log link function in a generalized linear model, the
transmission coefficient (number of infected birds in one day by one
infectious bird) is estimated.

\begin{Shaded}
\begin{Highlighting}[]
\NormalTok{sir.data[[}\DecValTok{1}\NormalTok{]]}\SpecialCharTok{$}\NormalTok{deltat }\OtherTok{\textless{}{-}} \DecValTok{1}
\NormalTok{fit}\OtherTok{\textless{}{-}}\FunctionTok{glm}\NormalTok{(}\FunctionTok{cbind}\NormalTok{(C, S}\SpecialCharTok{{-}}\NormalTok{ C)}\SpecialCharTok{\textasciitilde{}}\NormalTok{Vaccinated ,}\AttributeTok{offset=} \FunctionTok{log}\NormalTok{(deltat}\SpecialCharTok{*}\NormalTok{I}\SpecialCharTok{/}\NormalTok{N),}\AttributeTok{family =} \FunctionTok{binomial}\NormalTok{(}\AttributeTok{link =} \StringTok{"cloglog"}\NormalTok{), }\AttributeTok{data =}\NormalTok{ sir.data[[}\DecValTok{1}\NormalTok{]])}
\FunctionTok{print}\NormalTok{(}\StringTok{"Summary of fit contains log of transmission coefficients"}\NormalTok{)}
\end{Highlighting}
\end{Shaded}

\begin{verbatim}
## [1] "Summary of fit contains log of transmission coefficients"
\end{verbatim}

\begin{Shaded}
\begin{Highlighting}[]
\FunctionTok{summary}\NormalTok{(fit)}
\end{Highlighting}
\end{Shaded}

\begin{verbatim}
## 
## Call:
## glm(formula = cbind(C, S - C) ~ Vaccinated, family = binomial(link = "cloglog"), 
##     data = sir.data[[1]], offset = log(deltat * I/N))
## 
## Deviance Residuals: 
##     Min       1Q   Median       3Q      Max  
## -2.2370  -0.2794   0.0000   0.0000   2.0757  
## 
## Coefficients:
##               Estimate Std. Error z value Pr(>|z|)    
## (Intercept)     0.9684     0.2962   3.270  0.00108 ** 
## VaccinatedYes  -5.5304     1.0427  -5.304 1.13e-07 ***
## ---
## Signif. codes:  0 '***' 0.001 '**' 0.01 '*' 0.05 '.' 0.1 ' ' 1
## 
## (Dispersion parameter for binomial family taken to be 1)
## 
##     Null deviance: 98.745  on 32  degrees of freedom
## Residual deviance: 17.568  on 31  degrees of freedom
## AIC: 24.851
## 
## Number of Fisher Scoring iterations: 6
\end{verbatim}

\begin{Shaded}
\begin{Highlighting}[]
\FunctionTok{print}\NormalTok{(}\StringTok{"Transformed"}\NormalTok{)}
\end{Highlighting}
\end{Shaded}

\begin{verbatim}
## [1] "Transformed"
\end{verbatim}

\begin{Shaded}
\begin{Highlighting}[]
\FunctionTok{exp}\NormalTok{(}\FunctionTok{cumsum}\NormalTok{(fit}\SpecialCharTok{$}\NormalTok{coefficients))}
\end{Highlighting}
\end{Shaded}

\begin{verbatim}
##   (Intercept) VaccinatedYes 
##    2.63370435    0.01044116
\end{verbatim}
\end{block}
\end{frame}

\begin{frame}{Reference}
\protect\hypertarget{reference}{}
Velthuis, A. G. J., Bouma, A., Katsma, W. E. A., Nodelijk, G., \& De
Jong, M. C. M. (2007). Design and analysis of small-scale transmission
experiments with animals. Epidemiology and Infection, 135(2), 202--217.
\url{https://doi.org/10.1017/S095026880600673X}
\end{frame}

\end{document}
